%%%%%%%%%%%%%%%%%%%%%%%%%%%%%%%%%%%%%%%%%
% Short Sectioned Assignment
% LaTeX Template
% Version 1.0 (5/5/12)
%
% This template has been downloaded from:
% http://www.LaTeXTemplates.com
%
% Original author:
% Frits Wenneker (http://www.howtotex.com)
%
% License:
% CC BY-NC-SA 3.0 (http://creativecommons.org/licenses/by-nc-sa/3.0/)
%
%%%%%%%%%%%%%%%%%%%%%%%%%%%%%%%%%%%%%%%%%

%----------------------------------------------------------------------------------------
%	PACKAGES AND OTHER DOCUMENT CONFIGURATIONS
%----------------------------------------------------------------------------------------

\documentclass[paper=a4, fontsize=11pt]{scrartcl} % A4 paper and 11pt font size

\usepackage[utf8]{inputenc}
\usepackage[T1]{fontenc} % Use 8-bit encoding that has 256 glyphs
\usepackage{fourier} % Use the Adobe Utopia font for the document - comment this line to return to the LaTeX default
\usepackage{courier}
\usepackage[english]{babel} % English language/hyphenation
\usepackage{amsmath,amsfonts,amsthm} % Math packages
\usepackage[square,sort,numbers]{natbib}
\usepackage{listings}

\usepackage{lipsum} % Used for inserting dummy 'Lorem ipsum' text into the template

\usepackage{sectsty} % Allows customizing section commands
\allsectionsfont{\centering \normalfont\scshape} % Make all sections centered, the default font and small caps

\usepackage{fancyhdr} % Custom headers and footers
\pagestyle{fancyplain} % Makes all pages in the document conform to the custom headers and footers
\fancyhead{} % No page header - if you want one, create it in the same way as the footers below
\fancyfoot[L]{} % Empty left footer
\fancyfoot[C]{} % Empty center footer
\fancyfoot[R]{\thepage} % Page numbering for right footer
\renewcommand{\headrulewidth}{0pt} % Remove header underlines
\renewcommand{\footrulewidth}{0pt} % Remove footer underlines
\setlength{\headheight}{13.6pt} % Customize the height of the header
\lstset{
	linewidth=0.95\linewidth,
	breaklines=true,
	numbers=left,
    basicstyle=\footnotesize\ttfamily,
	numberstyle=\tiny,
	escapeinside={//*}{\^^M},
	mathescape=true
}

%----------------------------------------------------------------------------------------
%	TITLE SECTION
%----------------------------------------------------------------------------------------

\newcommand{\horrule}[1]{\rule{\linewidth}{#1}} % Create horizontal rule command with 1 argument of height

\title{	
\normalfont \normalsize 
\textsc{Università della Svizzera Italiana} \\ [25pt] % Your university, school and/or department name(s)
\horrule{0.5pt} \\[0.4cm] % Thin top horizontal rule
\huge Meeting - 28.03.2014 \\ % The assignment title
\horrule{2pt} \\[0.5cm] % Thick bottom horizontal rule
}

\author{Simon Maurer} % Your name

\date{\normalsize\today} % Today's date or a custom date

\begin{document}

\maketitle % Print the title

%----------------------------------------------------------------------------------------
%	PROBLEM 1
%----------------------------------------------------------------------------------------

\section{Choice of Device}
A reasonable board: Nexys 4 Artix-7 FPGA Board
\footnote{http://www.xilinx.com/products/boards-and-kits/1-3YZNP5.htm}
Having a look at the requirements for the exponential function on a FPGA:
\begin{itemize}
	\item \emph{double precision: 2024 slices + 3 DSP slices} on a Artix-7 100 T
		FPGA from Xilinx, there is enough space for 7 exp functions in parallel.
		One exp function can be fully pipelined \cite{RSSI08_Pottathuparambil}.
	\item \emph{single precision: 522 slices + ? DSP slices} on a Artix-7 100 T
		FPGA from Xilinx, there is enough space for 30 exp functions in
		parallel. One exp function can be fully pipelined \cite{ICFPT05_Detrey}.
\end{itemize}

The algorithm exposes $ N^2 $ parallel exponential functions. Assuming
$ N=100 $, for maximal performance a total amount of $ 10000 $ parallel
exponential functions are needed (with pipelining this number can be reduced at
a performance cost). No available FPGA can handle this amount of exp functions
in parallel.

\section{Extended Forward Algorithm}
Initialization of the forward algorithm. The variables $ dk $ and $ ok $ need
to be read from memory at each step  $ k $ (not shown in the code below).
\lstinputlisting[language=Octave]{../accelerator/model/forward_s.m}
By focusing on one step of the forward algorithm, it is important to note, that
the extended forward algorithm differs only in the computation of the transition
probabilities (function call in line 12) in comparison to the normal forward
algorithm.
\lstinputlisting[language=Octave]{../accelerator/model/forward_step_s.m}
Following the most expensive part of the algorithm, due to the exponential
function: the computation of the extended transition probabilities. The
extended transition probabilities depend on the chosen kernels. Here only one
kernel has been chosen (Gaussian cumulative distribution function).
\lstinputlisting[language=Octave]{../accelerator/model/compute_v.m}
During the meeting, the following decision has been taken: First focus only on
the implementation of the forward algorithm (by omitting the computation of
the extended transition probabilities) in FPGA.

\section{Computation Intense Functions}
A reference for computational intense functions has been named: (James M.)
Krause.

\section{Precision}
Precision is a very important parameter in order to create an efficient
accelerator. For the moment it is very difficult to define a necessary
precision to get reasonable results. After the implementation of the forward
algorithm on a FPGA it may be easier to discuss this problematic.

\bibliographystyle{siam}
\bibliography{biblio}

\end{document}
