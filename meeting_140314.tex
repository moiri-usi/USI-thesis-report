%%%%%%%%%%%%%%%%%%%%%%%%%%%%%%%%%%%%%%%%%
% Short Sectioned Assignment
% LaTeX Template
% Version 1.0 (5/5/12)
%
% This template has been downloaded from:
% http://www.LaTeXTemplates.com
%
% Original author:
% Frits Wenneker (http://www.howtotex.com)
%
% License:
% CC BY-NC-SA 3.0 (http://creativecommons.org/licenses/by-nc-sa/3.0/)
%
%%%%%%%%%%%%%%%%%%%%%%%%%%%%%%%%%%%%%%%%%

%----------------------------------------------------------------------------------------
%	PACKAGES AND OTHER DOCUMENT CONFIGURATIONS
%----------------------------------------------------------------------------------------

\documentclass[paper=a4, fontsize=11pt]{scrartcl} % A4 paper and 11pt font size

\usepackage[utf8]{inputenc}
\usepackage[T1]{fontenc} % Use 8-bit encoding that has 256 glyphs
\usepackage{fourier} % Use the Adobe Utopia font for the document - comment this line to return to the LaTeX default
\usepackage[english]{babel} % English language/hyphenation
\usepackage{amsmath,amsfonts,amsthm} % Math packages

\usepackage{lipsum} % Used for inserting dummy 'Lorem ipsum' text into the template

\usepackage{sectsty} % Allows customizing section commands
\allsectionsfont{\centering \normalfont\scshape} % Make all sections centered, the default font and small caps

\usepackage{fancyhdr} % Custom headers and footers
\pagestyle{fancyplain} % Makes all pages in the document conform to the custom headers and footers
\fancyhead{} % No page header - if you want one, create it in the same way as the footers below
\fancyfoot[L]{} % Empty left footer
\fancyfoot[C]{} % Empty center footer
\fancyfoot[R]{\thepage} % Page numbering for right footer
\renewcommand{\headrulewidth}{0pt} % Remove header underlines
\renewcommand{\footrulewidth}{0pt} % Remove footer underlines
\setlength{\headheight}{13.6pt} % Customize the height of the header

%----------------------------------------------------------------------------------------
%	TITLE SECTION
%----------------------------------------------------------------------------------------

\newcommand{\horrule}[1]{\rule{\linewidth}{#1}} % Create horizontal rule command with 1 argument of height

\title{	
\normalfont \normalsize 
\textsc{Università della Svizzera Italiana} \\ [25pt] % Your university, school and/or department name(s)
\horrule{0.5pt} \\[0.4cm] % Thin top horizontal rule
\huge Meeting - 14.03.2014 \\ % The assignment title
\horrule{2pt} \\[0.5cm] % Thick bottom horizontal rule
}

\author{Simon Maurer} % Your name

\date{\normalsize\today} % Today's date or a custom date

\begin{document}

\maketitle % Print the title

%----------------------------------------------------------------------------------------
%	PROBLEM 1
%----------------------------------------------------------------------------------------

\section{Problem Statement, Title and Abstract}
The document "problemStatement.pdf" presents the complete problem statement of
the master thesis. Remarks, suggestions and corrections are appreciated.

Concerning the title of the master thesis I have no better proposition as
"Accelerator for Event-based Failure Prediction". I think this title should be
improved but at the moment I cannot com up with something better. A good start
would be to give the algorithm of Felix a catching name.

I was asked to write an abstract in order to upload something to the ALaRI web
page. The final abstract I will write only at the very end of the work, in
order to also being able to mention the achievements and results. For now
I propose the following:

\begin{quotation}
In today's live it becomes increasingly important, that computer systems are
dependable. The reason being, that computer systems are used more and more in
areas where the failure of such a system can lead to catastrophic events.

In the event of a system failure it is of course desirable to fix the system as
soon as possible in order to minimize the downtime of the system (maximize the
availability). This can be accomplished by using different types of recovery
techniques, e.g. Check-pointing (create checkpoints to roll back/forward),
system replication (switch to a redundant system), fail over (reboot). All these
techniques require a certain amount of time to complete the recovery process,
time that is very expensive. In order to minimize this time, techniques have
been developed to anticipate upcoming failures.

The work of this master thesis consists in designing a hardware accelerator for
such a failure prediction algorithm. The algorithm in question has excellent
prediction capabilities but needs a lot of computation power and hence
computation time. The goal of the resulting accelerator will remedy this
problem and offer an exact and fast solution for predicting failures.
\end{quotation}

\section{Arguments Justifying the Work}
The email of Felix left some doubts to whether the acceleration of the
algorithm is useful. The following list will give some arguments to justify
the work.
\begin{description}
    \item[Too many parameters to be identified, estimated and set] \hfill \\
        Considering an embedded system, this is usually not a problem because
        the parameters are defined during the design phase and will never be
        changed afterwards.
    \item[Limited performance scalability] \hfill \\
		There are studies available claiming otherwise. The discussion of
		Neumanns work will provide some arguments against this statement.
    \item[Industry trends point towards cloud] \hfill \\
		In embedded systems it will still be beneficial to predict failures of
		single nodes. It is however important to keep the power and
		computational footprint low. This will be one of the major challenges.
		On the other hand, I think it would also be possible to also use this
		algorithm to monitor a distributed system and predict failures. It is
		only a matter of getting defining the events to feed to the algorithm.
\end{description}

\section{Discussion of Neumanns Work}
By only considering the computation complexity of the forward algorithm,
a theoretical speedup of factor N should be possible. This number is of course
limited by the parallelization overhead, the available computation units and
the data flow. In his work,

Following some thoughts about the implementation of Neumann:
\begin{itemize}
	\item implementation of classical forward algorithm, not the adapted version
		proposed by Felix. The adaptation requires more memory access and more
		computation steps but does not change the complexity order.
	\item high number of states compared to the work of Felix
	\item other work achieved better results on older GPUs (but with ignoring
		scaling). I believe this has been achieved by using a matrix
		multiplication approach (memory optimization).
	\item limited parallelization but optimization potential in data flow and
		single operation optimizations (FPGA)
	\item FPGA may be a better solution if also power consumption is taken into
		account. However due to a lower clock speed the speedup may not improve
		or even be smaller.
\end{itemize}
\end{document}
