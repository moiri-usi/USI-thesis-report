%%%%%%%%%%%%%%%%%%%%%%%%%%%%%%%%%%%%%%%%%
% Short Sectioned Assignment
% LaTeX Template
% Version 1.0 (5/5/12)
%
% This template has been downloaded from:
% http://www.LaTeXTemplates.com
%
% Original author:
% Frits Wenneker (http://www.howtotex.com)
%
% License:
% CC BY-NC-SA 3.0 (http://creativecommons.org/licenses/by-nc-sa/3.0/)
%
%%%%%%%%%%%%%%%%%%%%%%%%%%%%%%%%%%%%%%%%%

%----------------------------------------------------------------------------------------
%   PACKAGES AND OTHER DOCUMENT CONFIGURATIONS
%----------------------------------------------------------------------------------------

\documentclass[paper=a4, fontsize=11pt]{scrartcl} % A4 paper and 11pt font size

\usepackage[utf8]{inputenc}
\usepackage[T1]{fontenc} % Use 8-bit encoding that has 256 glyphs
\usepackage{fourier} % Use the Adobe Utopia font for the document - comment this line to return to the LaTeX default
\usepackage{courier}
\usepackage[english]{babel} % English language/hyphenation
\usepackage{amsmath,amsfonts,amsthm} % Math packages
\usepackage[square,sort,numbers]{natbib}
\usepackage{listings}

\usepackage{lipsum} % Used for inserting dummy 'Lorem ipsum' text into the template

\usepackage{sectsty} % Allows customizing section commands
\allsectionsfont{\centering \normalfont\scshape} % Make all sections centered, the default font and small caps

\usepackage{fancyhdr} % Custom headers and footers
\pagestyle{fancyplain} % Makes all pages in the document conform to the custom headers and footers
\fancyhead{} % No page header - if you want one, create it in the same way as the footers below
\fancyfoot[L]{} % Empty left footer
\fancyfoot[C]{} % Empty center footer
\fancyfoot[R]{\thepage} % Page numbering for right footer
\renewcommand{\headrulewidth}{0pt} % Remove header underlines
\renewcommand{\footrulewidth}{0pt} % Remove footer underlines
\setlength{\headheight}{13.6pt} % Customize the height of the header
\lstset{
    linewidth=0.95\linewidth,
    breaklines=true,
    numbers=left,
    basicstyle=\footnotesize\ttfamily,
    numberstyle=\tiny,
    escapeinside={//*}{\^^M},
    mathescape=true
}

%----------------------------------------------------------------------------------------
%   TITLE SECTION
%----------------------------------------------------------------------------------------

\newcommand{\horrule}[1]{\rule{\linewidth}{#1}} % Create horizontal rule command with 1 argument of height

\title{ 
\normalfont \normalsize 
\textsc{Università della Svizzera Italiana} \\ [25pt] % Your university, school and/or department name(s)
\horrule{0.5pt} \\[0.4cm] % Thin top horizontal rule
\huge Meeting - 23.05.2014 \\ % The assignment title
\horrule{2pt} \\[0.5cm] % Thick bottom horizontal rule
}

\author{Simon Maurer} % Your name

\date{\normalsize\today} % Today's date or a custom date

\begin{document}

\maketitle % Print the title

%----------------------------------------------------------------------------------------
%   PROBLEM 1
%----------------------------------------------------------------------------------------

\section{Architecture}
How does the architecture look like:
\begin{itemize}
    \item big pipeline to reduce memory access and parallelize computation of likelihood
    \item pipeline each stage of the big pipeline in order to maximize MACC throughput
    \item propose two solutions for the stages of the big pipeline:
    \begin{itemize}
        \item serial multiplication (register to DSP slice ratio is high) -> use ram to
            save intermediate steps to reduce register usage?
        \item parallel multiplication (register to DSP slice ratio is low) -> needs a
            powerful memory hierarchy to actually make sense (or maybe hardcode the
            parameters into a rom?)
    \end{itemize}
    \item use a controller to govern the pipelines
\end{itemize}

\section{Scaling and Precision}
Some important pointis about scaling:
\begin{itemize}
    \item neumann uses division on each element. Better: calculate
        a scaling factor with division operation and multiply with each element
    \item does not influence the order of complexity, has however a big
        impact on necessary recources.
    \item tradeoff: use scaling (more recources) lower precision needed (less recources)
        / no scaling (less recources) higher precision needed (more recources)
    \item if scaling, then use different method than the proposed one:
    \begin{itemize}
        \item avoid division (use shift right)
        \item avoid log function for likelihood (compute likelihood by accumulation and
            shift left. ATTENTION: overflow)
        \item either use average calculation to shift a fixed amount (check for overflow)
            or use the comparation element of the DSP slice.
    \end{itemize}
\end{itemize}

and about precision (floating point vs fixed point):
\begin{itemize}
    \item floating points are complicated
    \item floating points need more recources
    \item floating points are error prone
    \item floating points are precise
    \item fixed points are less precise (for a given width)
    \item fixed points are faster
    \item fixed points are easier to debug
\end{itemize}
-> use fixed point and simple scaling. Give precision ranges and design the system
with higher operant width.

\section{Testing}
\begin{itemize}
    \item use Nexis4 board
        \footnote{http://www.xilinx.com/products/boards-and-kits/1-3YZNP5.htm}
    \item store parameters and bitstream into flash, then load parameters into
        ram
    \item check memory solutions on the market to increase the data throughput
        and increase the speed of the whole system (cf. parallelized pipline
        stage)
    \item provide sequence stream either by loading a testsample into flash or
        by using an external interface (ethernet, serial port, etc)
\end{itemize}

\end{document}
