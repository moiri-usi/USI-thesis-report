\documentclass[mscthesis]{usiinfthesis}
\usepackage{lipsum}
\usepackage{listings}

\lstdefinelanguage{algebra}
{morekeywords={import,sort,constructors,observers,transformers,axioms,if,
else,end},
sensitive=false,
morecomment=[l]{//s},
}



\title{Accelerator for Event-based Failure Prediction} %compulsory
\specialization{Embedded Systems Design}%optional
\subtitle{Subtitle} %optional 
\author{Simon Maurer} %compulsory
\begin{committee}
    \advisor{Prof.}{Miroslaw}{Malek} %compulsory
    %\coadvisor{Prof.}{Student's}{Co-Advisor}{} %optional
\end{committee}
\Day{29.} %compulsory
\Month{Janaury} %compulsory
\Year{2014} %compulsory, put only the year
\place{Lugano} %compulsory

%\dedication{To my beloved} %optional
%\openepigraph{Someone said \dots}{Someone} %optional

%\makeindex %optional, also comment out \theindex at the end

\begin{document}

\maketitle %generates the titlepage, this is FIXED

\frontmatter %generates the frontmatter, this is FIXED

\begin{abstract}
\end{abstract}

%\begin{abstract}[Zusammenfassung]
%optional, use only if your external advisor requires it in his/er
%language 
%\\
%
%\lipsum
%\end{abstract}

\begin{acknowledgements}
\lipsum 
\end{acknowledgements}

\tableofcontents 
\listoffigures %optional
\listoftables %optional

\mainmatter

%===============================================================================
\chapter{Introduction}
%-------------------------------------------------------------------------------
\section{Problem Statement}
%-------------------------------------------------------------------------------
\section{Motivation}

The email of Felix left some doubts to whether the acceleration of the
algorithm is useful. The following list will give some arguments to justify
the work.
\begin{description}
    \item[Too many parameters to be identified, estimated and set] \hfill \\
        Considering an embedded system, this is usually not a problem because
        the parameters are defined during the design phase and will never be
        changed afterwards.
    \item[Limited performance scalability] \hfill \\
        There are studies available claiming otherwise. The discussion of
        Neumanns work will provide some arguments against this statement.
    \item[Industry trends point towards cloud] \hfill \\
        In embedded systems it will still be beneficial to predict failures of
        single nodes. It is however important to keep the power and
        computational footprint low. This will be one of the major challenges.
        On the other hand, I think it would also be possible to also use this
        algorithm to monitor a distributed system and predict failures. It is
        only a matter of getting defining the events to feed to the algorithm.
\end{description}

%-------------------------------------------------------------------------------
\section{Contributions}
%-------------------------------------------------------------------------------
\section{Document Structure}

%===============================================================================
\chapter{State of the Art}
%-------------------------------------------------------------------------------
\section{Failure Prediction}
%-------------------------------------------------------------------------------
\section{Accelerator}

%===============================================================================
\chapter{Implementation}
%-------------------------------------------------------------------------------
\section{Data Processing}
%-------------------------------------------------------------------------------
\section{Model Training}
%-------------------------------------------------------------------------------
\section{Sequence Processing}
%-------------------------------------------------------------------------------
\section{Classification}

%===============================================================================
\chapter{Acceleration}
Challanges of the acceleration
\begin{itemize}
    \item implementation of exp and log function (LUT, Taylor, ...)
    \item floating points vs fixed points
    \item precision
    \item choice of accelerator (Type, Model)
    \item find avaliable options for parallelization
\end{itemize}

Ideas on how to accelerate the online part of the algorithm
\begin{itemize}
    \item use high speed multiplier-accumulator (MAC) devices on a FPGA
    \item use MACs only on integer numbers and compute FP later manually
    \item minimize division (compute scaling factor once and then multiply)
    \item if precision allows use pipelining to precompute the factor
        $ b(j, o[k]) * v(i, j, k) $
    \item precompute known factors and store them in order to simlify online
        computation (e.g. parts of the kernel, classification thershold, ...).
    \item \dots
\end{itemize}

Possible optimizations of the algorithm:
\begin{itemize}
    \item use a regularization term in the cost function to prevent overfitting
    \item incorporate the offline part of the algorithm into the online part in
        order to deal with model aging
    \item \dots
\end{itemize}
%-------------------------------------------------------------------------------
\section{Theoretical Analysis}
%------------------------------------------
\subsection{Theory Base}
This section provides a brief overview of the computational steps done by the
proposed algorithm. The following description should provide a complete
blueprint of the adapted forward algorithm, that allows to implement it, but
without any explications or proofs related to the formulation. To be able to
understand the formal expression of the algorithm, first a definition of the
used parameters is provided.
\begin{itemize}
    \item N: number of states
    \item M: number of observation symbols
    \item L: observation sequence length
    \item R: number of cumulative probability distributions (kernels)
\end{itemize}
The delay of the event at time $ t_k $ with respect to the event at time
$ t_{k-1} $ is described as
\begin{equation}
    d_k = t_k-t_{k-1}
\end{equation}
One part of the algorithm is the model training. This part is not described
here. The features to be trained by the model training are however important
in this context because they are used by the adapted forward algorithm.
Following the features:
\begin{itemize}
    \item $ \pi_i $, forming the initial state probability vector
        $ \boldsymbol{\pi} $ of size $ N $
    \item $ b_i(o_j) $, forming the emission probability matrix $ B $ of size
        $ N \times M $
    \item $ p_{ij} $, forming the matrix of limiting transmission probabilities
        $ P $ of size $ N \times N $
    \item $ \omega_{ij, r} $, the weights of the kernel $ r $
    \item $ \theta_{ij, r} $, the parameters of the kernel $ r $
\end{itemize}
The adapted forward algorithm is defined as follows:
\begin{equation}
    \label{eq:forward_init}
    \alpha_0(i) = \pi_{i}b_{s_i}(O_0) \\
\end{equation}
\begin{equation}
    \label{eq:forward}
    \alpha_k(j) = \sum_{i=1}^{N} \alpha_{k-1}(i) v_{ij}(d_k) b_{s_j}(O_k);
    \quad 1 \leq k \leq L
\end{equation}
where
\begin{equation}
    \label{eq:V}
    v_{ij}(d_k) = \left\{
        \begin{array}{l l}
            p_{ij} d_{ij}(d_k)
                & \quad \text{if $j \neq i$}\\
            1 - \sum\limits_{\substack{h=1 \\ h \neq i}}^{N} p_{ih} d_{ih}(d_k)
                & \quad \text{if $j = i$}
        \end{array} \right.
\end{equation}
with
\begin{equation}
    \label{eq:D}
    d_{ij}(d_k) = \sum_{r=1}^{R} \omega_{ij,r}\kappa_{ij,r}(d_k|\theta_{ij, r})
\end{equation}
forming the matrix of cumulative transition duration distribution functions
$ D(d_k) $ of size $ N \times N \times L $.

For simplification reasons, only one kernel is used. Due to this, the kernel
weights can be ignored. Equation \ref{eq:D} can then be simplified:
\begin{equation}
    \label{eq:D_fact}
    d_{ij}(d_k) = \kappa_{ij}(d_k | \theta_{ij})
\end{equation}
Choosing the gaussian cumulative distribution results in the kernel parameters
$ \mu_{ij} $ and $ \sigma_{ij} $:
\begin{equation}
    \label{eq:kernel}
    \kappa_{ij, gauss}(d_k | \mu_{ij}, \sigma_{ij}) = 
        0.5\bigg [1 + \frac{2}{\pi}\Big (
            1 - \exp \big (-\frac{d_k - \mu_{ij}}{\sqrt 2 \sigma_{ij}}\big )
        \Big ) \bigg ]
\end{equation}
To prevent $ \alpha $ from going to zero very fast, at each step of the forward
algorithm a scaling is performed:
\begin{equation}
    \alpha_k(i) = c_k \alpha_k(i)
\end{equation}
with
\begin{equation}
    c_k = \frac{1}{\sum\limits_{i=1}^{N} \alpha_k(i)}
\end{equation}
then the sequence log-likelihood is computed:
\begin{equation}
    \log(P(\boldsymbol{o}|\lambda)) = -\sum\limits_{k=1}^{L} \log c_k
\end{equation}
where $ \lambda = \{\boldsymbol{\pi}, P, B, D(d_k) \} $,  and finally the
classification is performed:
\begin{equation}
    \label{eq:class}
    \text{class}(s) = F \iff \max_{i=1}^{u} \big [
        \log P(\boldsymbol{s}|\lambda_i)
    \big ] - \log P(\boldsymbol{s}|\lambda_0) > \log \theta
\end{equation}
with
\begin{equation}
    \label{eq:class_thresh}
    \theta = \frac{(r_{\bar{F}F} - r_{\bar{F}\bar{F}})P(c_{\bar{F}})}
        {(r_{F \bar{F}} - r_{FF})P(c_{F})}
\end{equation}
To calculate $ \theta $, the following parameters need to be set:
\begin{itemize}
    \item $ P(c_{\bar{F}}) $: prior of non-failure class
    \item $ P(c_F) $: prior of failure class
    \item $ r_{\bar{F}\bar{F}} $: true negative prediction
    \item $ r_{FF} $: true positive prediction
    \item $ r_{\bar{F}F} $: false positive prediction
    \item $ r_{F\bar{F}} $: false negative prediction
\end{itemize}

%------------------------------------------
\subsection{Pre-Computation of Known Parameters}
To reduce the computation effort of the algorithm during runtime, it may be
desirable to pre-compute factors composed of known parameters and store them in
the ROM of the accelerator. This may increase the necessary ROM on the
accelerator but reduce the number of costly computations.

Considering this, the equation \ref{eq:kernel} can be rewritten in the
following matter, by using the exponential properties:
\begin{equation}
    \label{eq:kernel_fact}
    \kappa_{ij, gauss}(d_k | \mu_{ij}, \sigma_{ij})
        = c_{ij, 1} - c_{ij, 2} \exp (c_{ij, 3} d_k)
\end{equation}
with the factors
\begin{equation}
    \label{eq:c1}
    c_{ij, 1} = 0.5 + \frac{1}{\pi}
\end{equation}
,
\begin{equation}
    \label{eq:c2}
    c_{ij, 2} = \frac{1}{\pi} \exp(\frac{\mu_{ij}}{\sqrt 2 \sigma_{ij}})
\end{equation}
and
\begin{equation}
    \label{eq:c3}
    c_{ij, 3} = - \frac{1}{\sqrt 2 \sigma_{ij}}
\end{equation}
By using the equations \ref{eq:D_fact} and \ref{eq:kernel_fact}, \ref{eq:V} can
now be written as
\begin{equation}
    \label{eq:V_fact}
    v_{ij}(d_k) = \left\{
        \begin{array}{l l}
            c_{ij, 1}^{'} - c_{ij, 2}^{'} \exp (c_{ij, 3} d_k)
                & \quad \text{if $j \neq i$}\\
            1 - \sum\limits_{\substack{h=1 \\ h \neq i}}^{N} \big [
                c_{ih, 1}^{'} - c_{ih, 2}^{'} \exp (c_{ih, 3} d_k) \big ]
                & \quad \text{if $j = i$}
        \end{array} \right.
\end{equation}
with the factors
\begin{equation}
    \label{eq:c1p}
    c_{ij, 1}^{'} = p_{ij} c_{ij, 1} = p_{ij} (0.5 + \frac{1}{\pi})
\end{equation}
and
\begin{equation}
    \label{eq:c2p}
    c_{ij, 2}^{'} = p_{ij} c_{ij, 2} = \frac{p_{ij}}{\pi}
        \exp(\frac{\mu_{ij}}{\sqrt 2 \sigma_{ij}})
\end{equation}
The factors \ref{eq:c1p}, \ref{eq:c2p} and \ref{eq:c3} can be precomputed and
stored in memory. Upon the computation of \ref{eq:V_fact} the factors are read
out of the ROM on the accelerator. Without pre-computation, the necessary ROM
storage to provide the model parameters on the accelerator is
\begin{equation}
    N \pi_{\#bit}
        + NM b_{\#bit}
        + N^2 ( p_{\#bit} + \mu_{\#bit} + \sigma_{\#bit})
        \,\text{bit}
\end{equation}
With the pre-computation described above
\begin{equation}
    N \pi_{\#bit}
        + NM b_{\#bit}
        + N^2 (c1_{\#bit}^{'} + c2_{\#bit}^{'} + c3_{\#bit})
        \,\text{bit}
\end{equation}
of available ROM storage is necessary.

\emph{Check speed and memory difference}

Also with equation \ref{eq:forward_init} pre-computing should be considered.
Instead of loading $ \pi $ into the ROM of the accelerator, a new
matrix $ B^{'} $ of size $ N \times M $ where
\begin{equation}
    b_{il}^{'} = \pi_i b_{s_i}(O_l)
\end{equation}
can be computed and stored. The matrix $ B $ also needs to be stored because
it is still used in equation \ref{eq:forward}. By using this pre-computation,
the necessary ROM storage is now:
\begin{equation}
    NM(b_{\#bit} + b_{\#bit}^{'})
        + N^2(c1_{\#bit}^{'} + c2_{\#bit}^{'} + c3_{\#bit})
        \,\text{bit}
\end{equation}

\emph{Check speed and memory difference}

The $ \log $ of equation \ref{eq:class_thresh} can also be precomputed as it is
used in equation \ref{eq:class} and all parameters are known before runtime.

%-------------------------------------------------------------------------------
\section{Model implementation}
%-------------------------------------------------------------------------------
\section{GPU-Acceleration}
%-------------------------------------------------------------------------------
\section{FPGA-Acceleration}

%===============================================================================
\chapter{Testing and Verification}
%-------------------------------------------------------------------------------
\section{Log Standard}
%-------------------------------------------------------------------------------
\section{Metrics}
%-------------------------------------------------------------------------------
\section{Automated Log Generation}
%-------------------------------------------------------------------------------
\section{Online Log Generation}

%===============================================================================
\chapter{Results}
%-------------------------------------------------------------------------------
\section{Speedup}
%-------------------------------------------------------------------------------
\section{Accuracy}

%===============================================================================
\chapter{Conclusion}
%-------------------------------------------------------------------------------
\section{Achievements}
%-------------------------------------------------------------------------------
\section{Future Work}

\nocite{*}

\appendix %optional, use only if you have an appendix

%===============================================================================
\chapter{Some material}
%\section{It's over\dots}

\backmatter

%\chapter{Glossary} %optional

%\bibliographystyle{alpha}
%\bibliographystyle{dcu}
%\bibliographystyle{plainnat}
%\bibliographystyle{plain}
%\bibliographystyle{abbrvnat}
\bibliographystyle{siam}
%\bibliographystyle{ieeetr}
\bibliography{biblio}

%\cleardoublepage
%\theindex %optional, use only if you have an index, must use
	  %\makeindex in the preamble

\end{document}
