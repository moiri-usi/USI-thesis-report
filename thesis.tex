\documentclass[mscthesis]{usiinfthesis}
\usepackage{lipsum}
\usepackage{listings}

\lstdefinelanguage{algebra}
{morekeywords={import,sort,constructors,observers,transformers,axioms,if,
else,end},
sensitive=false,
morecomment=[l]{//s},
}



\title{Accelerator for Event-based Failure Prediction} %compulsory
\specialization{Embedded Systems Design}%optional
\subtitle{Subtitle} %optional 
\author{Simon Maurer} %compulsory
\begin{committee}
    \advisor{Prof.}{Miroslaw}{Malek} %compulsory
    %\coadvisor{Prof.}{Student's}{Co-Advisor}{} %optional
\end{committee}
\Day{29.} %compulsory
\Month{Janaury} %compulsory
\Year{2014} %compulsory, put only the year
\place{Lugano} %compulsory

%\dedication{To my beloved} %optional
%\openepigraph{Someone said \dots}{Someone} %optional

%\makeindex %optional, also comment out \theindex at the end

\begin{document}

\maketitle %generates the titlepage, this is FIXED

\frontmatter %generates the frontmatter, this is FIXED

\begin{abstract}

\lipsum
\end{abstract}

%\begin{abstract}[Zusammenfassung]
%optional, use only if your external advisor requires it in his/er
%language 
%\\
%
%\lipsum
%\end{abstract}

\begin{acknowledgements}
\lipsum 
\end{acknowledgements}

\tableofcontents 
\listoffigures %optional
\listoftables %optional

\mainmatter

\chapter{Introduction}
\section{Problem Statement}
\section{Motivation}
\section{Contributions}
\section{Document Structure}

\chapter{State of the Art}
\section{Failure Prediction}
\section{Accelerator - SVM}

\chapter{Implementation}
\section{Data Processing}
\section{Model Training}
\section{Sequence Processing}
\section{Classification}

\chapter{Acceleration}
\section{Theoretical Analysis}
\section{Model implementation}
\section{GPU-Acceleration}
\section{FPGA-Acceleration}

\chapter{Testing and Verification}
\section{Log Standard}
\section{Metrics}
\section{Automated Log Generation}
\section{Online Log Generation}

\chapter{Results}
\section{Speedup}
\section{Accuracy}

\chapter{Conclusion}
\section{Achievements}
\section{Future Work}

%\textbf{Theorem 1 (Residue Theorem).}
%Let $f$ be analytic in the region $G$ except for the isolated singularities $a_1,a_2,\ldots,a_m$. If $\gamma$ is a closed rectifiable curve in $G$ which does not pass through any of the points $a_k$ and if $\gamma\approx 0$ in $G$ then
%\[
%\frac{1}{2\pi i}\int_\gamma f = \sum_{k=1}^m n(\gamma;a_k) \text{Res}(f;a_k).
%\]
%\textbf{Theorem 2 (Maximum Modulus).}
%\emph{Let $G$ be a bounded open set in $\mathbb{C}$ and suppose that $f$ is a continuous function on $G^-$ which is analytic in $G$. Then}
%\[
%\max\{|f(z)|:z\in G^-\}=\max \{|f(z)|:z\in \partial G \}.
%\]

%\lipsum \texttt{Some Test}
%\lstset{language=algebra,linewidth=0.95\linewidth,breaklines=true,numbers=left,
%basicstyle=\ttfamily,numberstyle=\tiny,escapeinside={//*}{\^^M},
%mathescape=true}
%\begin{lstlisting}
%import IntSpec, ItemSpec;
%
%sort cart; //*\label{sort}
%
%constructors //*\label{begin-sig}
%create() $\longrightarrow$ cart;
%insert(cart, item) $\longrightarrow$ cart;
%observers
%amount(cart) $\longrightarrow$ int;
%transformers
%delete(cart, item) $\longrightarrow$ cart; //*\label{end-sig}
%
%axioms //*\label{begin-axioms}
%forall c: cart, i, j: item 
%
%amount(create()) $=$ 0; //*\label{begin-amount}
%amount(insert(c,i)) $=$ amount(c) $+$ price(i); //*\label{end-amount}
%delete(create(),i) $=$ create(); //*\label{begin-delete}
%delete(insert(c,i),j) $=$
%if (i =$\:$= j) c
%else insert(delete(c,j),i); //*\label{end-axioms}
%end
%\end{lstlisting}

%As you can easily see from the above listing \citet{bbggs:iet07}
%define something weird based on the BPEL specification
%\citet{bpelspec}.
\nocite{*}

\appendix %optional, use only if you have an appendix

\chapter{Some material}
%\section{It's over\dots}

\backmatter

%\chapter{Glossary} %optional

%\bibliographystyle{alpha}
%\bibliographystyle{dcu}
%\bibliographystyle{plainnat}
%\bibliographystyle{plain}
%\bibliographystyle{abbrvnat}
\bibliographystyle{siam}
%\bibliographystyle{ieeetr}
\bibliography{biblio}

%\cleardoublepage
%\theindex %optional, use only if you have an index, must use
	  %\makeindex in the preamble

\end{document}
